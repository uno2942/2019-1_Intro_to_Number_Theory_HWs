%%%%%%%%%%%%%%%%%%%%%%%%%%%%%%%%%%%%%%%%%%%%%%%%%%%%%%%%%%%%%%%%%%%%%%%%%%%%%%%%%%%%
%Do not alter this block of commands.  If you're proficient at LaTeX, you may include additional packages, create macros, etc. immediately below this block of commands, but make sure to NOT alter the header, margin, and comment settings here. 
\documentclass[12pt]{article}
 \usepackage[margin=1in]{geometry} 
\usepackage{amsmath,amsthm,amssymb,amsfonts, enumitem, fancyhdr, color, comment, graphicx, environ, kotex, mathrsfs, mathtools, physics, esint, bm}
\pagestyle{fancy}
\setlength{\headheight}{65pt}
\newenvironment{problem}[2][Problem]{\begin{trivlist}
\item[\hskip \labelsep {\bfseries #1}\hskip \labelsep {\bfseries #2.}]}{\end{trivlist}}
\newenvironment{sol}
    {\emph{Solution:}
    }
    {
    \qed
    }
\specialcomment{com}{ \color{blue} \textbf{Comment:} }{\color{black}} %for instructor comments while grading
\NewEnviron{probscore}{\marginpar{ \color{blue} \tiny Problem Score: \BODY \color{black} }}

\DeclareMathOperator{\cc}{\mathbb{C}}
\DeclareMathOperator{\rr}{\mathbb{R}}
\DeclareMathOperator{\bA}{\mathbb{A}}
\DeclareMathOperator{\zz}{\mathbb{Z}}
\DeclareMathOperator{\fra}{\mathfrak{a}}
\DeclareMathOperator{\frb}{\mathfrak{b}}
\DeclareMathOperator{\frm}{\mathfrak{m}}
\DeclareMathOperator{\frp}{\mathfrak{p}}
\DeclareMathOperator{\slin}{\mathfrak{sl}}
\DeclareMathOperator{\Lie}{\mathsf{Lie}}
\DeclareMathOperator{\Alg}{\mathsf{Alg}}
\DeclareMathOperator{\Spec}{\mathrm{Spec}}
\DeclareMathOperator{\End}{\mathrm{End}}
\DeclareMathOperator{\rad}{\mathrm{rad}}
\newcommand*\Laplace{\mathop{}\!\mathbin\bigtriangleup}
\newcommand{\id}{\mathrm{id}}
\newcommand{\Hom}{\mathrm{Hom}}
\newcommand{\Sch}{\mathbf{Sch}}
\newcommand{\Ring}{\mathbf{Ring}}
\newcommand{\T}{\mathcal{T}}
\newcommand{\B}{\mathcal{B}}
\newcommand{\Mod}[1]{\ (\mathrm{mod}\ #1)}
\newtheorem{lemma}{Lemma}
\newtheorem{theorem}{Theorem}
\newtheorem{proposition}{Proposition}
%%%%%%%%%%%%%%%%%%%%%%%%%%%%%%%%%%%%%%%%%%%%%%%%%%%%%%%%%%%%%%%%%%%%%%%%%%%%%%%%%





%%%%%%%%%%%%%%%%%%%%%%%%%%%%%%%%%%%%%%%%%%%%%
%Fill in the appropriate information below
\lhead{SungBin Park, 20150462}  %replace with your name
\rhead{Intro. to Number Theory \\ 예습 위주 \\ Assignment 2} %replace XYZ with the homework course number, semester (e.g. ``Spring 2019"), and assignment number.
%%%%%%%%%%%%%%%%%%%%%%%%%%%%%%%%%%%%%%%%%%%%%


%%%%%%%%%%%%%%%%%%%%%%%%%%%%%%%%%%%%%%
%Do not alter this block.
\begin{document}
%%%%%%%%%%%%%%%%%%%%%%%%%%%%%%%%%%%%%%


%Solutions to problems go below.  Please follow the guidelines from https://www.overleaf.com/read/sfbcjxcgsnsk/


%Copy the following block of text for each problem in the assignment.
\section{Assign 2}
\begin{problem}{1}
\end{problem}
\begin{enumerate}

    \item[(a)] Assume that $aa_i\equiv aa_j \mod n$ for some $i\neq j$, then $a(a_i-a_j)\equiv 0 \mod n$. Since $(a,n)=1$, $n\mid a_i-a_j$. However, $0<\abs{a_i-a_j}<n$ for $\{i,j\}\subset \{1, 2, \ldots, n\}$, which is a contradiction. Therefore, $aa_i$ and $aa_j$ are not congruent in modulo $n$. By the pigeonhole principle, $\{1,2,3,\ldots,n\}=\{a_1, a_2,\ldots, a_n\}=\{aa_1, aa_2, \ldots, aa_n\}$ as a complete set of residues modulo $n$.

    \item[(b)] I'll prove the statement: If $\{a_1, a_2, \ldots, a_{\phi(n)}\}$ is a reduced set of residues modulo $n$, so is $\{aa_1, aa_2, \ldots, aa_{\phi(n)}\}$.
    \begin{proof}
    By the same reason of (a), $aa_i$ and $aa_j$ are not congruent in modulo $n$. Furthermore, we can readily see that $(aa_i, n)=1$ since $(a,n)=1\Rightarrow \exists x_1,y_1~ax_1+ny_1=1$ and $(a_i, n)=1\Rightarrow \exists x_2, y_2~a_ix_2+ny_2=1$, so $aa_i(x_1x_2)+n(a_ix_2y_1+ax_1y_2+ny_1y_2)=1$. By the definition of $\phi(n)$, $\{aa_1, aa_2, \ldots, aa_{\phi(n)}\}$ forms a reduced set of residues modulo $n$.
    \end{proof} 
    
\end{enumerate}

%Copy the following block of text for each problem in the assignment.
\begin{problem}{2}
\end{problem}
I'll show that $8n+3$, $n\in \mathbb{Z}$ is the solution set. We can easily see that this is a subset of the solution of $3x\equiv 9 \mod 24$. Conversely, assume that there exists $n\in \mathbb{Z}$ and $a\in\{0, 1, 2, \hat{3}, 4, \ldots, 7\}$ such that $8n+a$ satisfies $3x\equiv 9 \mod 24$. Then, it means $3a\equiv 9 \mod 24$, but since $0\leq 3a\leq 21$, $3a$ is not congruent to $9$ in modulo $24$,which is contradiction. Therefore, $8n+3$, $n\in \mathbb{Z}$ is the solution set.
\begin{problem}{3}
\end{problem}
Since $(9,7)=(9,5)=(7,5)=1$, I can use Chinese remainder theorem.

$35\equiv -1 \mod 9$. $45\equiv 3 \mod 7$. $63 \equiv 3 \mod 5$. $8\cdot 35+4\cdot45+1\cdot63+315n$, $n\in \mathbb{Z}$ is the solution.

\begin{problem}{4}
\end{problem}
\begin{enumerate}
\item[(i)] For modulo $4$, $x^2$ is congruent to $0$ or $1$, so $x^2+y^2$ is congruent to $0,1,2$. However, $103\equiv 3 \mod 4$. Therefore, there is no solution for $x^2+y^2=103$.
\item[(ii)] For modulo $8$, $x^2$ is congruent to $0, 4, 1$. Therefore, $x^2+y^2+z^2$ congruents to $0, 1, 2, 4, 5, 6$. However, $95\equiv 7 \mod 8$. Therefore, $x^2+y^2+z^2=95$ does not have solution.
\end{enumerate}
%Copy the following block of text for each problem in the assignment.
\newpage
\begin{problem}{5}
\end{problem}


First, assume that $x$ is odd number. Then, $x$ is form of $4n+1$ or $4n+3$. For $4n+1$, $x^3\equiv 1 \mod 4$ an for $4n+3$, $x^3\equiv 3 \mod 4$. For either case, it can not satisfies $x^3\equiv(x-1)^3+(x-2)^3+(x-3)^3 \mod 4$. Therefore, $x$ should be even number.

Assume that $x$ is of form $4n$. Then, $(x-1)^3\equiv 12n-1 \mod 16$, $(x-3)^3\equiv 12(n-1)+1 \mod 16$, and $(x-2)^3\equiv 8 \mod 16$. Therefore, $0\equiv 8n-4 \mod 16$, which is impossible. Hence, $x$ is of form $4n+2$.

Let $f(x)=(x-1)^3+(x-2)^3+(x-3)^3-x^3=2x^3-18x^2+42x-36$. For $x\geq 10$, $x^2(2x-18)+42x-36>0$. Also for $x\leq -2$, $x^2(2x-18)+42x-36<0$. For $x=2$, $f(x)=-8$. Therefore, the only solution in integer is $f(6)=0$.


%Copy the following block of text for each problem in the assignment.
\begin{problem}{6}
\end{problem}
In $(\mathbb{Z}/10\mathbb{Z})^*$, which is multiplicative group of $\mathbb{Z}/10\mathbb{Z}$, $7^{139}\equiv7\cdot 49^{69}\equiv 7\cdot (-1)^{69}\equiv 3 \mod 10$.

$13^{2018}\equiv 3^{2018}\equiv (-1)^{1009}\equiv 9 \mod 10$. Therefore, the last digit of $7^{139}$ an $13^{2018}$ are $3$ and $9$.

\begin{problem}{7}
\end{problem}
Computing $x^2 \mod 23$,
\begin{equation*}
\begin{split}
    0^2&\equiv0 \\
    1^2&\equiv1 \\
    2^2&\equiv4 \\
    3^2&\equiv9 \\
    4^2&\equiv16 \\
    5^2&\equiv2 \\
    6^2&\equiv13 \\
    7^2&\equiv3 \\
    8^2&\equiv18 \\
    9^2&\equiv12 \\
    10^2&\equiv8 \\
    11^2&\equiv6 \\
\end{split}
\end{equation*}
and by the symmetricity, $(23-n)^2\equiv n^2 \mod 23$ for $n\in\{1, 2,\ldots, 11\}$. There is no $22$ in $x^2 \mod 23$. Therefore, there is no solution for $x^2+1=23y$.


%Copy the following block of text for each problem in the assignment.
\newpage
\begin{problem}{8}
\end{problem}
I'll prove this by induction. Assume that the statement is true for $k\geq 3$. Then, it means that for odd number satisfying $1\leq a< 2^{k}$, $a^{2^{k-2}}=n\cdot 2^k+1$ for some $n$, so $a^{2^{k-1}}=(n\cdot 2^k+1)^2$ and $a^2\equiv 1 \mod 2^{n+1}$ for $1\leq a< 2^{k}$. For $2^k<a<2^{k+1}$, we know that $(2^{k+1}-a)^2\equiv a^2 \mod 2^{k+1}$ reducing $1\leq a< 2^{k}$ case. Therefore, $a^{2^{k-2}}\equiv 1 \mod 2^k$ for all $k\geq 3$.
\newpage
\section{Quiz 3}
\begin{problem}{1}
\end{problem}
Since $2^9\equiv 512\equiv -(2^7+1) \mod 641$, $512^3\cdot 2^5\equiv -(2^7+1)^3\cdot 2^5 \equiv -2(2^9+2^2)^2(2^7+1)\equiv -2(-129+4)^2(2^7+1)\equiv-2(2^7-3)^2(2^7+1)\equiv -(2^{15}-3\cdot 2^9+18)(2^7+1)\equiv -((-2^7-1)2^6+3(129)+18)(2^7+1)\equiv -( 2^4(2^7+1)+341)(2^7+1)\equiv -(2^2(-2^7-1)+357)(2^7+1)\equiv-(2^7+354)(2^7+1)\equiv -1 \mod 641 $. Therefore, $641\mid 2^{2^5}+1$.

\begin{problem}{2}
\end{problem}
First, I'll show that any combination is an integer. To prove this, I'll use Pascal's triangle:
\begin{equation*}
    \binom{n}{k}=\frac{n!}{(n-k)!k!}=\frac{(n-1)!}{(n-1-k)!k!}+\frac{(n-1)!}{(n-k)!(k-1)!}=\binom{n-1}{k}+\binom{n-1}{k-1}
\end{equation*}
For induction, assume that $\binom{n}{k}\in\mathbb{Z}$ for $0\leq k\leq n$. (The base case $n=0,1$ can be easily shown.) For $n+1$ case with $1\leq k\leq n$,
\begin{equation*}
    \binom{n+1}{k}=\binom{n}{k}+\binom{n}{k-1} \in\mathbb{Z}
\end{equation*}
For $k=0,n+1$, it is also an integer. Therefore, $\binom{n}{k}\in\mathbb{Z}$ for all $n$ and $0\leq k\leq n$.
\begin{enumerate}
    \item[(i)] Since $n<p$, there is no prime factor of $p$ in $n!$, but $(2n)!$ have prime factor $p$. Therefore, $\binom{2n}{n}\equiv 0 \mod p$
    \item[(ii)] Again, no prime factor of $p$ in $n!$, so $\binom{2n}{n}\equiv 0 \mod p^2$ means $p^2\mid (2n)!$. If $p\mid (2n)!/p$, there exists $a\neq p$ such that $a\in\{1,2,\ldots, 2n\}$ and $p\mid a$. However, it means that $2n< 2p\leq a$, which is contradiction. Therefore, $\binom{2n}{n}$ is not congruent to $0$ in modulo $p^2$.
\end{enumerate}
\begin{problem}{3}
\end{problem}
\begin{enumerate}
    \item [(i)] By Wilson's theorem, $(p-1)!\equiv -1 \mod p$. Since $p$ is odd and $(p-a)\equiv -a \mod p$ for $0\leq a\leq p$, \begin{equation*}
    (-1)^{\frac{p-1}{2}}1^2 \cdot 2^2 \cdots \left(\frac{p-1}{2}\right)^2\equiv -1 \mod p
\end{equation*}
Since $2^{p-1}\equiv 1 \mod p$,
\begin{equation*}
    2^{p-1}\left(1^2\cdot 2^2 \cdots \left(\frac{p-1}{2}\right)^2\right)\equiv 2^2\cdot 4^2\cdots (p-1)^2\equiv (-1)^{\frac{p+1}{2}} \mod p
\end{equation*}
    \item[(ii)] By Wilson's theorem, $\left((p-1)!\right)^2\equiv 1 \mod p$ and we already shown that $2^2\cdot 4^2\cdots (p-1)^2\equiv (-1)^{\frac{p+1}{2}} \mod p$. Therefore, $1^2\cdot 3^2\cdots (p-2)^2\equiv 2^2\cdot 4^2\cdots (p-1)^2 \mod p$. (Note that $\mathbb{Z}/10\mathbb{Z}$ is a field.)
\end{enumerate}
\newpage
\begin{problem}{4}
For $m=63$, The solution contains $1,4,16,22$.
\end{problem}
\end{document}
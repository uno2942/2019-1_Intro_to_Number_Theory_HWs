%%%%%%%%%%%%%%%%%%%%%%%%%%%%%%%%%%%%%%%%%%%%%%%%%%%%%%%%%%%%%%%%%%%%%%%%%%%%%%%%%%%%
%Do not alter this block of commands.  If you're proficient at LaTeX, you may include additional packages, create macros, etc. immediately below this block of commands, but make sure to NOT alter the header, margin, and comment settings here. 
\documentclass[12pt]{article}
 \usepackage[margin=1in]{geometry} 
\usepackage{amsmath,amsthm,amssymb,amsfonts, enumitem, fancyhdr, color, comment, graphicx, environ, kotex, mathrsfs, mathtools, physics, esint, bm}
\pagestyle{fancy}
\setlength{\headheight}{65pt}
\newenvironment{problem}[2][Problem]{\begin{trivlist}
\item[\hskip \labelsep {\bfseries #1}\hskip \labelsep {\bfseries #2.}]}{\end{trivlist}}
\newenvironment{sol}
    {\emph{Solution:}
    }
    {
    \qed
    }
\specialcomment{com}{ \color{blue} \textbf{Comment:} }{\color{black}} %for instructor comments while grading
\NewEnviron{probscore}{\marginpar{ \color{blue} \tiny Problem Score: \BODY \color{black} }}

\DeclareMathOperator{\cc}{\mathbb{C}}
\DeclareMathOperator{\rr}{\mathbb{R}}
\DeclareMathOperator{\bA}{\mathbb{A}}
\DeclareMathOperator{\zz}{\mathbb{Z}}
\DeclareMathOperator{\fra}{\mathfrak{a}}
\DeclareMathOperator{\frb}{\mathfrak{b}}
\DeclareMathOperator{\frm}{\mathfrak{m}}
\DeclareMathOperator{\frp}{\mathfrak{p}}
\DeclareMathOperator{\slin}{\mathfrak{sl}}
\DeclareMathOperator{\Lie}{\mathsf{Lie}}
\DeclareMathOperator{\Alg}{\mathsf{Alg}}
\DeclareMathOperator{\Spec}{\mathrm{Spec}}
\DeclareMathOperator{\End}{\mathrm{End}}
\DeclareMathOperator{\rad}{\mathrm{rad}}
\newcommand*\Laplace{\mathop{}\!\mathbin\bigtriangleup}
\newcommand{\id}{\mathrm{id}}
\newcommand{\Hom}{\mathrm{Hom}}
\newcommand{\Sch}{\mathbf{Sch}}
\newcommand{\Ring}{\mathbf{Ring}}
\newcommand{\T}{\mathcal{T}}
\newcommand{\B}{\mathcal{B}}
\newcommand{\Mod}[1]{\ (\mathrm{mod}\ #1)}
\newtheorem{lemma}{Lemma}
\newtheorem{theorem}{Theorem}
\newtheorem{proposition}{Proposition}
%%%%%%%%%%%%%%%%%%%%%%%%%%%%%%%%%%%%%%%%%%%%%%%%%%%%%%%%%%%%%%%%%%%%%%%%%%%%%%%%%





%%%%%%%%%%%%%%%%%%%%%%%%%%%%%%%%%%%%%%%%%%%%%
%Fill in the appropriate information below
\lhead{SungBin Park, 20150462}  %replace with your name
\rhead{Intro. to Number Theory \\ 예습 위주 \\ Assignment 2} %replace XYZ with the homework course number, semester (e.g. ``Spring 2019"), and assignment number.
%%%%%%%%%%%%%%%%%%%%%%%%%%%%%%%%%%%%%%%%%%%%%


%%%%%%%%%%%%%%%%%%%%%%%%%%%%%%%%%%%%%%
%Do not alter this block.
\begin{document}
%%%%%%%%%%%%%%%%%%%%%%%%%%%%%%%%%%%%%%


%Solutions to problems go below.  Please follow the guidelines from https://www.overleaf.com/read/sfbcjxcgsnsk/


%Copy the following block of text for each problem in the assignment.
\begin{problem}{1}
\end{problem}
\begin{enumerate}

    \item[(a)] 

    \item[(b)] I'll prove the statement: If $\{a_1, a_2, \ldots, a_{\phi(n)}\}$ is a reduced set of residues modulo $n$, so is $\{aa_1, aa_2, \ldots, aa_{\phi(n)}\}$.
    \begin{proof}
    Assume that $aa_i\equiv aa_j \mod n$ for some $i\neq j$, then $a(a_i-a_j)\equiv 0 \mod n$. Since $(a,n)=1$, $n\equiv a_i-a_j$. However, $0<\abs{a_i-a_j}<n$ for $\{i,j\}\subset \{1, 2, \ldots, \phi(n)\}$, which is a contradiction. Therefore, $aa_i$ and $aa_j$ are not congruent in modulo $n$. Furthermore, we can readily see that $(aa_i, n)=1$ since $(a,n)=1\Rightarrow \exists x_1,y_1~ax_1+ny_1=1$ and $(a_i, n)=1\Rightarrow \exists x_2, y_2~a_ix_2+ny_2=1$, so $aa_i(x_1x_2)+n(a_ix_2y_1+ax_1y_2+ny_1y_2)=1$. By the definition of $\phi(n)$, $\{aa_1, aa_2, \ldots, aa_{\phi(n)}\}$ forms a reduced set of residues modulo $n$.
    \end{proof} 
    
\end{enumerate}



\newpage
%Copy the following block of text for each problem in the assignment.
\begin{problem}{2}
\end{problem}
I'll show that $8n+3$, $n\in \mathbb{Z}$ is the solution set. We can easily see that this is a subset of the solution of $3x\equiv 9 (\mod 24)$. Conversely, assume that there exists $n\in \mathbb{Z}$ and $a\in\{0, 1, 2, 4, \ldots, 7\}$ such that $8n+a$ satisfies $3x\equiv 9 (\mod 24)$. Then, it means $3a\equiv 9 (\mod 24)$, but since $0\geq 3a\geq 21$, $3a$ is not congruent to $9$ in modulo $24$,which is contradiction. Therefore, $8n+3$, $n\in \mathbb{Z}$ is the solution set.
\begin{problem}{3}
\end{problem}
Since $(9,7)=(9,5)=(7,5)=1$, I can use Chinese remainder theorem.

$35\equiv -1 (\mod 9)$. $45\equiv 3 (\mod 7)$. $63 \equiv 3 (\mod 5)$. $8*35+4*45+1*63+315n$ is the solution.

\begin{problem}{4}
\end{problem}
\begin{enumerate}
\item[(i)] For modulo $4$, $x^2$ is congruent to $0$ or $1$, so $x^2+y^2$ is congruent to $0,1,2$. However, $103\equiv 4 (\mod 4)$. Therefore, there is no solution for $x^2+y^2=103$.
\item[(ii)] For modulo $8$, $x^2$ is congruent to $0, 4, 1$. Therefore, $x^2+y^2+z^2$ congruents to $0, 1, 2, 4, 5, 6$. However, $95\equiv 7 (\mod 8)$. Therefore, $x^2+y^2+z^2=95$ does not have solution.
\end{enumerate}
%Copy the following block of text for each problem in the assignment.
\begin{problem}{5}
\end{problem}
Assume that there are only finite primes of the form $4n+3$, and such primes be $4n_i+3$ for $i\in \{1, 2, \ldots, m\}$ in increasing order. Consider $a=2\left(\prod_{i=1}^m (4n_i+3)\right)+1$. As an odd natural number, it has prime decomposition, and every prime factor can not be of the form $4n_i+3$ since $a \equiv 1\mod (4n_i+3)$. Also, $a \equiv 3\mod 4$, but all the multiple of primes of the form $4n+1$ is $1$, which is contradiction .Therefore, There should be another prime of the form $4n+3$, which is contradiction to finiteness of primes of the form $4n+3$, and it shows there are infinitely many primes of the form $4n+3$.




%Copy the following block of text for each problem in the assignment.
\begin{problem}{6}
\end{problem}
Consider a number $a=(n!)^2+1$ for a fixed $n$. I'll show that the minimal prime factor of $a$, will be denoted by $p$, is of form $4n+1$. If $p\leq n$, then $p\mid a\Rightarrow p\mid 1$, which is non-sense, so $p>n$. Since $p\mid a$, $(n!)^2\equiv -1\mod p$ and $(n!^2)^{\frac{p-1}{2}}\equiv (-1)^{\frac{p-1}{2}} \mod p$ since $(\zz/p\zz)^*$ has a ring structure. ($(\zz/p\zz)^*$ is a multiplicative group of $\zz/p\zz$.) Also, note that $p\nmid (n!)^2\Rightarrow p\nmid n!$.) Since $p\nmid n!$, by the Fermat's little theorem, $(n!)^{p-1}\equiv 1 \mod p$ and it implies $p \equiv 1 \mod 4$.
Finally, assume that there are finitely many primes of the form $4n+1$ and let the set $\{4n_i+1\}_{i=1}^m$ in increasing order. Then, the smallest prime factor of $((4n_m+1)!)^2+1$ is a prime of the form $4n+1$ which is bigger than $4n_m+1$, which is contradiction. Therefore, there are infinitely many primes of the form $4n+1$.

The proof of Fermat's little theorem: We know that $\abs{(\zz/p\zz)^*}=p-1$, so for $a\in \mathbb{N}$ satisfying $p\nmid a$, $a^{p-1}\equiv 1 \mod p$.

%Copy the following block of text for each problem in the assignment.
\begin{problem}{7}
\end{problem}
We can readily show that $\frac{1}{n}\geq \int_{n-1}^{n} \frac{1}{x} dx$ for all $n>1$ since $\frac{1}{x}$ is non-increasing continuous function on $\rr^+$. Therefore, for all $N\in \mathbb{N}$, $\sum\limits_{i=1}^N \frac{1}{i}\geq \int_1^{N+1}\frac{1}{x} dx=\ln(N+1)$. Therefore, $\sum\limits_{i=1}^N \frac{1}{i}\rightarrow \infty$ as $N\rightarrow \infty$.


%Copy the following block of text for each problem in the assignment.
\newpage
\begin{problem}{8}
\end{problem}
I'll enumerator each prime $\{p_i\}$ in increasing order. For fixed $n>1$, we know that $\ln \left(\sum\limits_{i=1}^n \frac{1}{i}\right)\geq \ln \ln N$ by problem 7. Also, for fixed $m\in \mathbb{N}$,
\begin{equation*}
\begin{split}
\sum\limits_{i=1}^m \ln \left(\sum\limits_{j=0}^\infty \frac{1}{p_i^j}\right)&=-\sum\limits_{i=1}^m \ln \left(1-\frac{1}{p_i}\right) \\
&\leq \sum\limits_{i=1}^\infty \frac{1}{p_i} + \sum\limits_{j=2}^\infty \left(\frac{1}{j}\sum\limits_{i=1}^\infty \frac{1}{p^j_i}\right)~(\textrm{By Taylor expansion}) \\
&\leq \sum\limits_{i=1}^\infty \frac{1}{p_i} + \sum\limits_{j=2}^\infty \frac{1}{j}\left(\left(\frac{1}{2^j}\right)+\sum\limits_{i=3}^\infty \frac{1}{i^j}\right) \\
&\leq \sum\limits_{i=1}^\infty \frac{1}{p_i} + 1+\left(\sum\limits_{j=2}^\infty \int_2^\infty \frac{1}{x^j} dx \right) \\
&\leq \sum\limits_{i=1}^\infty \frac{1}{p_i} + 1+\left(\sum\limits_{j=2}^\infty \frac{1}{j-1}\frac{1}{2^{j-1}}\right) \\
&\leq \sum\limits_{i=1}^\infty \frac{1}{p_i} + 2
\end{split}
\end{equation*}
All the rational number of the form $\frac{1}{n}$ can be generated by $\sum\limits_{i=1}^\infty\left(\sum\limits_{j=0}^\infty \frac{1}{p_i^j}\right)$ by Euler's argument. Therefore, $\sum\limits_{i=1}^m\ln\left(\sum\limits_{j=0}^\infty \frac{1}{p_i^j}\right)\rightarrow \infty$ as $m\rightarrow \infty$, but if $\sum\limits_{i=1}^\infty \frac{1}{p_i}$ is bounded, it makes contradiction. Therefore, $\sum\limits_{i=1}^\infty \frac{1}{p_i}$ diverges.
\end{document}
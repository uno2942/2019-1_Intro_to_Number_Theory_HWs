\documentclass{article}
\usepackage{graphicx, amssymb}
\usepackage{amsmath}
\usepackage{amsfonts}
\usepackage{amsthm}
\usepackage{kotex}
\usepackage{bm}
\usepackage{hyperref}
\usepackage{xcolor}
\usepackage{mathrsfs}
\usepackage{mathtools}
\usepackage{physics}
\usepackage{ esint }

\textwidth 6.5 truein 
\oddsidemargin 0 truein 
\evensidemargin -0.50 truein 
\topmargin -.5 truein 
\textheight 8.5in

\DeclareMathOperator{\cc}{\mathbb{C}}
\DeclareMathOperator{\rr}{\mathbb{R}}
\DeclareMathOperator{\bA}{\mathbb{A}}
\DeclareMathOperator{\zz}{\mathbb{Z}}
\DeclareMathOperator{\fra}{\mathfrak{a}}
\DeclareMathOperator{\frb}{\mathfrak{b}}
\DeclareMathOperator{\frm}{\mathfrak{m}}
\DeclareMathOperator{\frp}{\mathfrak{p}}
\DeclareMathOperator{\slin}{\mathfrak{sl}}
\DeclareMathOperator{\Lie}{\mathsf{Lie}}
\DeclareMathOperator{\Alg}{\mathsf{Alg}}
\DeclareMathOperator{\Spec}{\mathrm{Spec}}
\DeclareMathOperator{\End}{\mathrm{End}}
\DeclareMathOperator{\rad}{\mathrm{rad}}
\newcommand*\Laplace{\mathop{}\!\mathbin\bigtriangleup}
\newcommand{\id}{\mathrm{id}}
\newcommand{\Hom}{\mathrm{Hom}}
\newcommand{\Sch}{\mathbf{Sch}}
\newcommand{\Ring}{\mathbf{Ring}}
\newcommand{\T}{\mathcal{T}}
\newcommand{\B}{\mathcal{B}}
\newcommand{\Mod}[1]{\ (\mathrm{mod}\ #1)}
\newtheorem{lemma}{Lemma}
\newtheorem{theorem}{Theorem}
\newtheorem{proposition}{Proposition}

\begin{document}


\title{Introduction to Number Theory - HW1}
\author{SungBin Park, Physics, 20150462} 

 \maketitle

\section*{Problem 1}
\begin{enumerate}
    \item[(a)] ($\Leftarrow$) Since $(a,b)=d$, $\exists x_0, y_0\in \zz$ such that $ax_0+by_0=d$. Since $d\mid n$, $\exists c\in \zz$ such that $dc=n$. Therefore, $acx_0+bcy_0=dc=n$ and the solution of $ax+by=n$ is $x=cx_0$, $y=cy_0$.
    
    ($\Rightarrow$) Assume $d\nmid n$. Then, $\exists r\in \mathbb{N}$ such that $0<r<d$ and $n=dq+r$ for some $q\in \mathbb{Z}$. If $ax+by=n$ is soluble and the solution is $ax_1+by_1=n$, then $a(x_1-qx_0)+b(y_1-qy_0)=r$ and it means $d$ is not g.c.d. which is contradiction. Therefore, $ax+by=n$ is not solution.
    \item[(b)] 
\end{enumerate}
\section*{Problem 2}

\section*{Problem 3}

\section*{Problem 4}

\section*{Problem 5}

\section*{Problem 6}

\section*{Problem 7}
\end{document}
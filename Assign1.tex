%%%%%%%%%%%%%%%%%%%%%%%%%%%%%%%%%%%%%%%%%%%%%%%%%%%%%%%%%%%%%%%%%%%%%%%%%%%%%%%%%%%%
%Do not alter this block of commands.  If you're proficient at LaTeX, you may include additional packages, create macros, etc. immediately below this block of commands, but make sure to NOT alter the header, margin, and comment settings here. 
\documentclass[12pt]{article}
 \usepackage[margin=1in]{geometry} 
\usepackage{amsmath,amsthm,amssymb,amsfonts, enumitem, fancyhdr, color, comment, graphicx, environ, kotex, mathrsfs, mathtools, physics, esint, bm}
\pagestyle{fancy}
\setlength{\headheight}{65pt}
\newenvironment{problem}[2][Problem]{\begin{trivlist}
\item[\hskip \labelsep {\bfseries #1}\hskip \labelsep {\bfseries #2.}]}{\end{trivlist}}
\newenvironment{sol}
    {\emph{Solution:}
    }
    {
    \qed
    }
\specialcomment{com}{ \color{blue} \textbf{Comment:} }{\color{black}} %for instructor comments while grading
\NewEnviron{probscore}{\marginpar{ \color{blue} \tiny Problem Score: \BODY \color{black} }}

\DeclareMathOperator{\cc}{\mathbb{C}}
\DeclareMathOperator{\rr}{\mathbb{R}}
\DeclareMathOperator{\bA}{\mathbb{A}}
\DeclareMathOperator{\zz}{\mathbb{Z}}
\DeclareMathOperator{\fra}{\mathfrak{a}}
\DeclareMathOperator{\frb}{\mathfrak{b}}
\DeclareMathOperator{\frm}{\mathfrak{m}}
\DeclareMathOperator{\frp}{\mathfrak{p}}
\DeclareMathOperator{\slin}{\mathfrak{sl}}
\DeclareMathOperator{\Lie}{\mathsf{Lie}}
\DeclareMathOperator{\Alg}{\mathsf{Alg}}
\DeclareMathOperator{\Spec}{\mathrm{Spec}}
\DeclareMathOperator{\End}{\mathrm{End}}
\DeclareMathOperator{\rad}{\mathrm{rad}}
\newcommand*\Laplace{\mathop{}\!\mathbin\bigtriangleup}
\newcommand{\id}{\mathrm{id}}
\newcommand{\Hom}{\mathrm{Hom}}
\newcommand{\Sch}{\mathbf{Sch}}
\newcommand{\Ring}{\mathbf{Ring}}
\newcommand{\T}{\mathcal{T}}
\newcommand{\B}{\mathcal{B}}
\newcommand{\Mod}[1]{\ (\mathrm{mod}\ #1)}
\newtheorem{lemma}{Lemma}
\newtheorem{theorem}{Theorem}
\newtheorem{proposition}{Proposition}
%%%%%%%%%%%%%%%%%%%%%%%%%%%%%%%%%%%%%%%%%%%%%%%%%%%%%%%%%%%%%%%%%%%%%%%%%%%%%%%%%





%%%%%%%%%%%%%%%%%%%%%%%%%%%%%%%%%%%%%%%%%%%%%
%Fill in the appropriate information below
\lhead{SungBin Park, 20150462}  %replace with your name
\rhead{Intro. to Number Theory \\ Assignment 1} %replace XYZ with the homework course number, semester (e.g. ``Spring 2019"), and assignment number.
%%%%%%%%%%%%%%%%%%%%%%%%%%%%%%%%%%%%%%%%%%%%%


%%%%%%%%%%%%%%%%%%%%%%%%%%%%%%%%%%%%%%
%Do not alter this block.
\begin{document}
%%%%%%%%%%%%%%%%%%%%%%%%%%%%%%%%%%%%%%


%Solutions to problems go below.  Please follow the guidelines from https://www.overleaf.com/read/sfbcjxcgsnsk/


%Copy the following block of text for each problem in the assignment.
\begin{problem}{1}
\end{problem}
\begin{enumerate}

    \item[(a)] ($\Leftarrow$) Since $(a,b)=d$, $\exists x_0, y_0\in \zz$ such that $ax_0+by_0=d$. Since $d\mid n$, $\exists c\in \zz$ such that $dc=n$. Therefore, $acx_0+bcy_0=dc=n$ and the solution of $ax+by=n$ is $x=cx_0$, $y=cy_0$.
    
    ($\Rightarrow$) Assume $d\nmid n$. Then, $\exists r\in \mathbb{N}$ such that $0<r<d$ and $n=dq+r$ for some $q\in \mathbb{Z}$. If $ax+by=n$ is soluble and the solution is $ax_1+by_1=n$, then $a(x_1-qx_0)+b(y_1-qy_0)=r$ and it means $d$ is not g.c.d. which is contradiction. Therefore, $ax+by=n$ is not solution.

    \item[(b)] Fix an arbitrary $m\in \mathbb{N}$. Define g.c.d. of $a_1, a_2, \ldots, a_m$ by the natural number $d$ satisfying:
    
    \begin{enumerate}
    	\item[(1)] The $d$ divides all $a_i$, i.e., $d \mid a_i$ for all $i\in \{1, 2, \ldots, m\}$.
    	\item[(2)] For any natural number satisfying (1), it divides $d$, i.e., if $e\in \mathbb{N}$ satisfies (1), $e\mid d$.
    \end{enumerate}
    
    By (1), g.c.d $d$d is a common divisor since it divides all $a_i$ and by (2), it is the greatest number among (1). Now. I need to show that g.c.d exists for all case.    
    Let $M=\{\sum\limits_{i=1}^m a_i x_i\in \mathbb{N} \mid x_i\in \mathbb{Z}\}$. Since $a\in M$, $M\neq \phi$ and by the Well-ordering principle, there exists $\min M\in \mathbb{N}$. I'll show that $\min M=d$.
    By the definition of $\min M$, there exists $\{x_i\}$ such that $\sum\limits_{i=1}^m a_i x_i=\min M$. WLOG, I'll show that $a_1$ is divisible by $\min M$. Assume $\min M \nmid a_1$, then there exists $q\in \zz$, $0<r<\min M$ such that $a_1=(\min M)q+r$ and $a_1-(\sum\limits_{i=1}^m a_i x_i)q=r<\min M$, which is contradiction to minimality of $\min M$. Therefore, $\min M \nmid a_i$ for all $i$. By (2) above, $\min M \mid d$.
    Since $d$ divides all $a_i$, conversely, $d\mid \min M$. Therefore, $d\mid \min M$. It show that g.c.d. always exists for all cases for all $m$.
    
    I'll denote the g.c.d. of $\{a_i\}_{i=1}^m$ by $(a_1, a_2, \ldots, a_m)$ or $d$.
    
    \item[(c)] ($\Leftarrow$) Since $(a_1, a_2, \ldots, a_m)=d$, $\exists x_i\in \zz$ such that $\sum\limits_{i=1}^m a_i x^0_i=d$. Since $d\mid n$, $\exists c\in \zz$ such that $dc=n$. Therefore, $\sum\limits_{i=1}^m a_i (cx_i)=dc=n$ and the solution of $\sum\limits_{i=1}^m a_i x_i=d$ is $x'_i=cx_i$.
    
    ($\Rightarrow$) Assume $d\nmid n$. Then, $\exists r\in \mathbb{N}$ such that $0<r<d$ and $n=dq+r$ for some $q\in \mathbb{Z}$. If $\sum\limits_{i=1}^m a_i x_i=n$ is soluble and the solution is $\{x'_i\}$, then $\sum\limits_{i=1}^m a_i(x'_i-qx^0_i)=r$ and it means $d$ is not g.c.d. which is contradiction. Therefore, $ax+by=n$ is not solution.
\end{enumerate}




%Copy the following block of text for each problem in the assignment.
\begin{problem}{2}
\end{problem}
\begin{enumerate}
\item $432=95\cdot 4+52$
\item $95=52\cdot 1+43$
\item $52=43\cdot 1+9$
\item $43=9\cdot 4+7$
\item $9=7 \cdot 1+2$
\item $7=2\cdot 3 +1$
\end{enumerate}
so,
\begin{enumerate}
\item $43=95-(432-95\cdot 4)=95\cdot 5-432$
\item $9=(432-95\cdot 4)-(95\cdot 5-432)=2\cdot 432 - 9\cdot 95$
\item $7=(5\cdot 95-432)-4\cdot (2\cdot 432 - 9\cdot 95)=-9\cdot 432 + 41\cdot 95$
\item $2=(2\cdot 432 - 9\cdot 95)-(-9\cdot 432 + 41\cdot 95)=11\cdot 432 - 50\cdot 95$.
\item $1=(-9\cdot 432 + 41\cdot 95)-3\cdot(11\cdot 432 - 50\cdot 95)=-42\cdot 432 + 191 \cdot 95$
\end{enumerate}



%Copy the following block of text for each problem in the assignment.
\begin{problem}{3}
\end{problem}
$35\cdot 3- 55\cdot 2=-5$. $46(35\cdot 3- 55\cdot 2)+3\cdot 77=138\cdot 35-92 \cdot 55+3\cdot 77=1$.



%Copy the following block of text for each problem in the assignment.
\begin{problem}{4}
\end{problem}
I'll show it by contradiction. Assume that $\sum\limits_{i=2}^n \frac{1}{i}$ is a natural number $q$. Since $n$ is bounded, there exists $N$ such that $2^N\leq n < 2^{N+1}$. Then,
\begin{equation*}
2^N\left(\frac{1}{2}+\cdots +\frac{1}{n}\right)=+2^{N-1}+\cdots+1+2^N\left(\frac{1}{3}+\frac{1}{5}+\cdots\right)=2^N q
\end{equation*}
(If $n$ is not form of $2^N$, the last term ends with $\frac{1}{n}$, and if not, $\frac{1}{n-1}$.)
Then,
\begin{equation*}
\frac{2^N q-(2^{N}-1)}{2^N}=\frac{1}{3}+\frac{1}{5}+\cdots
\end{equation*}
Since $(2^N q-(2^{N}-1), 2^N)=1$, the RHS should have $2^N$ term in denominator, but each terms in RHS has at most $2^{N-1}$ term, which is contradiction. Therefore, $\sum\limits_{i=2}^n \frac{1}{i}$ is not a natural number for all $n$.




%Copy the following block of text for each problem in the assignment.
\begin{problem}{5}
\end{problem}
Assume that there are only finite primes of the form $4n+3$, and such primes be $4n_i+3$ for $i\in \{1, 2, \ldots, m\}$ in increasing order. Consider $a=2\left(\prod_{i=1}^m 4n_i+3\right)+1$. As an odd natural number, it has prime decomposition, and every prime factor can not be of the form $4n_i+3$ since $a \equiv 1\mod 4n_i+3$. Also, $a \equiv 3\mod 4$ but all the multiple of primes of the form $4n+1$ is $1$, which is contradiction .Therefore, There should be another prime of the form $4n+3$, which is contradiction to finiteness of primes of the form $4n+3$, and it shows there are infinitely many primes of the form $4n+3$.




%Copy the following block of text for each problem in the assignment.
\begin{problem}{6}
\end{problem}
Consider a number $a=(n!)^2+1$ for a fixed $n$. I'll show that the minimal prime factor, I'll denote it $p$, of $a$ is of the form $4n+1$. If $p\leq n$, $p\mid a\Rightarrow p\mid 1$, which is non-sense, so $p>n$. Since $p\mid a$, $(n!)^2\equiv -1\mod p$ and $(n!^2)^{\frac{p-1}{2}}\equiv (-1)^{\frac{p-1}{2}} \mod p$ since $(\zz/p\zz)^*$ has a ring structure. ($(\zz/p\zz)^*$ is a multiplicative group of $\zz/p\zz$. Also, note that $p\nmid (n!)^2\Rightarrow p\nmid n!$.) Since $p\nmid n!$, by the Fermat's little theorem, $(n!)^{p-1}\equiv 1 \mod p$ and it implies $p \equiv 1 \mod 4$.
Finally, assume that there are finitely many primes of the form $4n+1$ and let the set $\{4n_i+1\}_{i=1}^m$ in increasing order. Then, the smallest prime factor of $((4n_m+1)!)^2+1$ is a prime of the form $4n+1$ which is bigger than $4n_m+1$, which is contradiction. Therefore, there are infinitely many primes of the form $4n+1$.




%Copy the following block of text for each problem in the assignment.
\begin{problem}{7}
\end{problem}
We can readily show that $\frac{1}{n}\geq \int_{n-1}^{n} \frac{1}{x} dx$ for all $n>1$ since $\frac{1}{x}$ is non-increasing function on $\rr^+$. Therefore, for all $N\in \mathbb{N}$, $\sum\limits_{i=1}^N \frac{1}{i}\geq \int_1^{N+1}\frac{1}{x} dx=\ln(N+1)$. Therefore, $\sum\limits_{i=1}^N \frac{1}{i}\rightarrow \infty$ as $N\rightarrow \infty$.



%Copy the following block of text for each problem in the assignment.
\begin{problem}{8}
\end{problem}
For fixed $n>1$, we know that $\ln \left(\sum\limits_{i=1}^n \frac{1}{i}\right)\geq \ln \ln N$ by problem 7. Also,
\begin{equation*}
\begin{split}
\sum\limits_{i=1}^m \ln \left(\sum\limits_{j=0}^\infty \frac{1}{p_i^j}\right)&=-\sum\limits_{i=1}^m \ln \left(1-\frac{1}{p_i}\right) \\
&\leq \sum\limits_{i=1}^\infty \frac{1}{p_i} + \sum\limits_{j=2}^\infty \left(\frac{1}{j}\sum\limits_{i=1}^\infty \frac{1}{p^j_i}\right)~(\textrm{By Taylor expansion}) \\
&\leq \sum\limits_{i=1}^\infty \frac{1}{p_i} + \sum\limits_{j=2}^\infty \frac{1}{j}\left(\left(\frac{1}{2^j}\right)+\sum\limits_{i=3}^\infty \frac{1}{i^j}\right) \\
&\leq \sum\limits_{i=1}^\infty \frac{1}{p_i} + 1+\left(\sum\limits_{j=2}^\infty \int_2^\infty \frac{1}{x^j} dx \right) \\
&\leq \sum\limits_{i=1}^\infty \frac{1}{p_i} + 1+\left(\sum\limits_{j=2}^\infty \frac{1}{j-1}\frac{1}{2^{j-1}}\right) \\
&\leq \sum\limits_{i=1}^\infty \frac{1}{p_i} + 2
\end{split}
\end{equation*}
Also, all the rational number of the form $\frac{1}{n}$ can be generated by $\sum\limits_{i=1}^\infty\left(\sum\limits_{j=0}^\infty \frac{1}{p_i^j}\right)$ by Euler's argument. Therefore, $\sum\limits_{i=1}^m\left(\sum\limits_{j=0}^\infty \frac{1}{p_i^j}\right)\rightarrow \infty$ as $m\rightarrow \infty$, but if $\sum\limits_{i=1}^\infty \frac{1}{p_i}$ is bounded, it makes contradiction. Therefore, $\sum\limits_{i=1}^\infty \frac{1}{p_i}$ diverges.


\end{document}
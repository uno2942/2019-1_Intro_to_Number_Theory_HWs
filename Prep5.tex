%%%%%%%%%%%%%%%%%%%%%%%%%%%%%%%%%%%%%%%%%%%%%%%%%%%%%%%%%%%%%%%%%%%%%%%%%%%%%%%%%%%%
%Do not alter this block of commands.  If you're proficient at LaTeX, you may include additional packages, create macros, etc. immediately below this block of commands, but make sure to NOT alter the header, margin, and comment settings here. 
\documentclass[12pt]{article}
 \usepackage[margin=1in]{geometry} 
\usepackage{amsmath,amsthm,amssymb,amsfonts, enumitem, fancyhdr, color, comment, graphicx, environ, kotex, mathrsfs, mathtools, physics, esint, bm}
\pagestyle{fancy}
\setlength{\headheight}{65pt}
\newenvironment{problem}[2][Problem]{\begin{trivlist}
\item[\hskip \labelsep {\bfseries #1}\hskip \labelsep {\bfseries #2.}]}{\end{trivlist}}
\newenvironment{sol}
    {\emph{Solution:}
    }
    {
    \qed
    }
\specialcomment{com}{ \color{blue} \textbf{Comment:} }{\color{black}} %for instructor comments while grading
\NewEnviron{probscore}{\marginpar{ \color{blue} \tiny Problem Score: \BODY \color{black} }}

\DeclareMathOperator{\cc}{\mathbb{C}}
\DeclareMathOperator{\rr}{\mathbb{R}}
\DeclareMathOperator{\bA}{\mathbb{A}}
\DeclareMathOperator{\zz}{\mathbb{Z}}
\DeclareMathOperator{\fra}{\mathfrak{a}}
\DeclareMathOperator{\frb}{\mathfrak{b}}
\DeclareMathOperator{\frm}{\mathfrak{m}}
\DeclareMathOperator{\frp}{\mathfrak{p}}
\DeclareMathOperator{\slin}{\mathfrak{sl}}
\DeclareMathOperator{\Lie}{\mathsf{Lie}}
\DeclareMathOperator{\Alg}{\mathsf{Alg}}
\DeclareMathOperator{\Spec}{\mathrm{Spec}}
\DeclareMathOperator{\End}{\mathrm{End}}
\DeclareMathOperator{\rad}{\mathrm{rad}}
\newcommand*\Laplace{\mathop{}\!\mathbin\bigtriangleup}
\newcommand{\id}{\mathrm{id}}
\newcommand{\Hom}{\mathrm{Hom}}
\newcommand{\Sch}{\mathbf{Sch}}
\newcommand{\Ring}{\mathbf{Ring}}
\newcommand{\T}{\mathcal{T}}
\newcommand{\B}{\mathcal{B}}
\newcommand{\Mod}[1]{\ (\mathrm{mod}\ #1)}
\newtheorem{lemma}{Lemma}
\newtheorem{theorem}{Theorem}
\newtheorem{proposition}{Proposition}
%%%%%%%%%%%%%%%%%%%%%%%%%%%%%%%%%%%%%%%%%%%%%%%%%%%%%%%%%%%%%%%%%%%%%%%%%%%%%%%%%





%%%%%%%%%%%%%%%%%%%%%%%%%%%%%%%%%%%%%%%%%%%%%
%Fill in the appropriate information below
\lhead{SungBin Park, 20150462}  %replace with your name
\rhead{Intro. to Number Theory \\ 예습 위주 \\ Preparation 5} %replace XYZ with the homework course number, semester (e.g. ``Spring 2019"), and assignment number.
%%%%%%%%%%%%%%%%%%%%%%%%%%%%%%%%%%%%%%%%%%%%%


%%%%%%%%%%%%%%%%%%%%%%%%%%%%%%%%%%%%%%
%Do not alter this block.
\begin{document}
%%%%%%%%%%%%%%%%%%%%%%%%%%%%%%%%%%%%%%


%Solutions to problems go below.  Please follow the guidelines from https://www.overleaf.com/read/sfbcjxcgsnsk/


%Copy the following block of text for each problem in the assignment.
\section{Prep. 5}
\begin{problem}{1}
\end{problem}
First, I'll claim that for $a$ and $d\mid p-1$ satisfying $d<p-1$ and $a^d\not\equiv 1 \mod p$ , $\sum\limits_{i=1}^{(p-1)/d} a^{id}\equiv 0\mod p$. This is because
\begin{equation*}
    \sum\limits_{i=0}^{(p-1)/d-1} a^{id}\equiv \frac{a^{p-1}-1}{a^d-1}\equiv 0 \mod p
\end{equation*}
Also, I'll use a result of binomial theorem:
\begin{equation*}
    (1-1)^n=\sum\limits_{i=0}^n \binom{n}{i} (-1)^n=0
\end{equation*}
Let $g$ be a primitive root of $p$ and let's start the proof from an equation:
\begin{equation*}
    \sum\limits_{i=1}^{p-1} g^i\equiv 0\mod p
\end{equation*}
Let's write the prime decomposition of $p-1$ by $q_1^{e_1}q_2^{e_2}\cdots q_n^{e_n}$. We know that if $0<j<p-1$, $g^j\not\equiv 1\mod p$, therefore we get
\begin{equation*}
    \sum\limits_{j=0}^{(p-1)/q_i} g^{jq_i}\equiv \equiv 0 \mod p
\end{equation*}
for each $i$. By subtracting each equation to original one, we can remove ...
for each prime factor $p_j$ of $p-1$, subtract $\sum\limits_{i=1}^{(p-1)/p_j} a^{id}\equiv 0\mod p$ on both side. After the step, we add $\sum\limits_{i=1}^{(p-1)/p_j p_k} a^{id}\equiv 0\mod p$ for $j\neq k$. Repeating this procedure for $d=\prod\limits_{l=1}^{m} p_l\leq p$, we can get the sum of primitive roots. Now, I'll divide the cases.
\begin{enumerate}
    \item If $p-1$ is not square and have odd prime factors, we substract $a^{p-1}\equiv 1\mod p$ at final step, making the sum of primitive roots congruents to $-1\mod p$.
    \item If $p-1$ is not square and have even prime factors, we add $a^{p-1}\equiv 1\mod p$ at final step, making the sum of primitive roots congruents to $1\mod p$.
    \item If $p-1$ is square, we don't add or subtract $a^{p-1}\equiv 1 \mod p$ at final step, making the sum of primitive roots congruents to $0\mod p$.
\end{enumerate}
Therefore, the sum of primitive roots is congruents to $\mu(p-1)\mod p$.
\begin{problem}{2}
\end{problem}
The possible number set of $y^2\mod 7$ is $\{0,1,2,4\}$ and of $5x^3\mod 7$ is $\{0,2,5\}$. Therefore, the possible solution set would be $(y,x)=\{(3,3),(3,5),(3,6),(4,3),(4,5),(4,6)\}$.
\begin{problem}{3}
\end{problem}
First, note that $p\nmid (p-1)!$, so if I show that $\left(1+\frac{1}{2}+\cdots+\frac{1}{p-1}\right)/p\equiv 0\mod p$, the fact that the numerator is divisible by $p^2$ easily follows.

Since
\begin{equation*}
    \begin{split}
    1+\frac{1}{2}+\cdots+\frac{1}{p-1} \\
    &\equiv \left(1+\frac{1}{p-1}\right) + \left(\frac{1}{2}+\frac{1}{p-2}\right) + \cdots + \left(\frac{1}{\frac{p-1}{2}}+\frac{1}{\frac{p+1}{2}}\right) \\
    &\equiv -p\left(\sum\limits_{i=1}^{\frac{p-1}{2}} \frac{1}{i^2}\right)\\
    &\equiv 0\mod p
    \end{split}
\end{equation*}
I only need to show that $\left(\sum\limits_{i=1}^{\frac{p-1}{2}} \frac{1}{i^2}\right)\equiv 0\mod p$, but
\begin{equation*}
    2\left(\sum\limits_{i=1}^{\frac{p-1}{2}} \frac{1}{i^2}\right)\equiv \left(\sum\limits_{i=1}^{p-1} \frac{1}{i^2}\right)\equiv \left(\sum\limits_{i=1}^{p-1} i^2\right)\equiv \frac{(p-1)p(2p-1)}{6}\equiv 0\mod p.
\end{equation*}
(note that $\zz/p\zz$ is a field, so there is unique inverse element for each element.) The last congruence follows from the fact that $p>3$. Therefore, the numerator of $\left(1+\frac{1}{2}+\cdots+\frac{1}{p-1}\right)/p\equiv 0\mod p$ is divisible by $p^2$.
\begin{problem}{4}
\end{problem}
Fix $a\in \{1, 2,\ldots, n\}$, then there exists $b\in \{1, 2,\ldots, n\}$ such that $(a,n)=b$. It implies that $\left(\frac{a}{b}, \frac{n}{b}\right)=1$. Therefore, $\sum\limits_{d\mid n}\phi(d)= n$: if $a$ is counted in $\phi(d)$, then we consider $ad$ and for any $a<n$, there exists $d=(a,n)$ and it contained in $\phi(d)$.
\begin{problem}{5}
\begin{enumerate}
    \item[(a)] We know that the order of $5$ in modulo $2^j$ should divide $\varphi(2^j)=2^{j-1}$. Also, for $d>3$,
    \begin{equation*}
        5^{2^d}-1 = (5-1)\prod_{i=0}^{d-1} (5^{2^i}+1)
    \end{equation*}
    and for $i>0$, $4\nmid 5^{2^i}+1$ since $5^{2^i}\equiv (4+1)^{2^i}\equiv 1\mod 4$. Therefore, $2^{d+2}\mid 5^{2^d}-1$, but $2^{d+3}\nmid 5^{2^d}-1$ and it means the order of $5$ in modulo $2^j$ is $j-2$.
    \item[(b)] Let $f:\{(-1)^l5^m\mid l\in\{0,1\},m\in\{0,1,2,\ldots, 2^{j-2}-1\}\}\rightarrow \{1,3,\ldots, 2^{j-1}\}$ by $f(x)=x$. What we want is to show that $f$ is bijective. We know that $5^m$ is distinct number for each $m\in\{0,1,2,\ldots, 2^{j-2}-1\}\}$, so I'll show that $\{5^m\mid m\in\{0,1,2,\ldots, 2^{j-2}-1\}\}\cap \{-5^m\mid m\in\{0,1,2,\ldots, 2^{j-2}-1\}\}=\phi$. However, $5\equiv 1 \mod 4$, so $5^m\equiv 1\mod 4$ and $-5^m\equiv 3\mod 4$ for all $m$. Therefore, $f$ is bijective and every odd integer $a$ is congruent to just one integer of the form $(-1)^l5^m$ in modulo $2^j$, where $l = \{0, 1\}$ and $m = \{0, 1, \ldots , 2^{j-2}-1\}$.
\end{enumerate}
\end{problem}
\begin{problem}{6}
\end{problem}
The possible set of $a$ satisfying $x^2\equiv a\mod 13$ is $\{1,4,9,3,12,10\}$, so $\left(\frac{a}{13}\right)=1$ for $a=1,3,4,9,10,12$ and $-1$ for the others.
\end{document}